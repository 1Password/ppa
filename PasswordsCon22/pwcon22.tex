% !TEX program = xelatex
\documentclass[xcolor={dvipsnames,table,hyperref}]{beamer}

% pick one. "hide notes" is default
\setbeameroption{hide notes}
% \setbeameroption{show notes}
% \setbeameroption{show notes on second screen = left}
% \setbeameroption{show only notes}
% \setbeameroption{show only slides with notes}

\usepackage{fontspec}
\usepackage{unicode-math}

% Use CM Bright for math calligraphic instead of Fira
\DeclareMathAlphabet{\mathcal}{OMS}{cmbrs}{m}{n}
\setmonofont{Fira Mono}

\usepackage{polyglossia}
\setdefaultlanguage{english}
\setotherlanguages{german}

\usepackage{xltxtra}
\usepackage[backend=biber,style=alphabetic,
  doi=true, url=true, isbn=false]{biblatex}

% When Jeffrey Goldberg builds this, he can use his bibliography database
% But he should include in the repository a .bib file that contains what
% is needed for these slides.
% He can do that by first generating the .bcf using his resources, but then
% using
%  biber --output_format=bibtex --output_resolve pwcon22.bcf
%  rm pwcon22.bcf
% to create a pwcon22_biber.bib file to be included.
%
\IfFileExists{./pwcon22_biber.bib}{%
  \addbibresource{pwcon22_biber.bib}
}{%
  \addbibresource{crypto.bib}
  \addbibresource{goldberg.bib}
}

\renewcommand*{\bibfont}{\tiny}

% Beamer (metropolis) style setup
\usepackage{FiraSans}
\usefonttheme{professionalfonts}
\usetheme{metropolis}
\definecolor{BitsBlue2}{HTML/hsb}{0A56BF/0.597,0.95,0.75} % guess % H=215
\metroset{titleformat = smallcaps,
  titleformat plain = regular,
  subsectionpage = progressbar,
  block = fill}
\setbeamercolor{palette primary}{bg=BitsBlue2}
\usepackage{appendixnumberbeamer}
% I give up trying to find a sans math. Fira Math is too heavy,
% and Fira Math Light is missing symbols.

\setbeamertemplate{theorems}[numbered]

\usepackage{stNotes}

% \mcomm for comments in lines of math
\providecommand*{\mcomm}[1]{\text{\footnotesize (#1)}}

\usepackage{epigraph}
\setlength{\epigraphwidth}{\textwidth}
\setlength{\epigraphrule}{0pt}

\author[J.~Goldberg]
{Jeffrey Goldberg\texorpdfstring{\\ \texttt{jeff@1Password.com}}{}}
\institute[1Password]{1Password}
\title{Can we safely learn about users' passwords}
\subtitle{What we should never know}
\date{August 9, 2020\\
\footnotesize{(Last revised: \today)}}

\hypersetup{colorlinks=true, allcolors=black, urlcolor=magenta}
\providecommand*{\reporoot}{https://gitlab.com/1Password/ppa}
\providecommand*{\sourceroot}{\reporoot/-/tree/main/2022-04-08}

\begin{document}
\maketitle

\section{Setting the scene}

\begin{frame}{Who am I}
  \begin{itemize}
    \item Jeffrey Goldberg
    \item Working at 1Password since 2010
    \item Wants to know everything
    \item So committed to Zero-Knowledge that I crave ignorance
  \end{itemize}
  \note[item]{If you notice a conflict between those last two items you are ready for this talk}

\end{frame}

\begin{frame}{1Password}
  \begin{itemize}
    \item A password manager. Client software and a service.
    \item The client software is where most of the action is.
          \begin{itemize}
            \item Unlocked client knows names of vaults and names of items. Server doesn't.
            \item Unlocked client knows password strengths. Server doesn't.
            \item Unlocked client has decrypted encryption keys. Server doesn't.
          \end{itemize}
    \item Designed so that we learn as little about users secrets as possible
    \item We like to think we know a thing or two about password behavior
  \end{itemize}
  \note[item]{I have to ration my nagging of our Marketing department. And so we are not sponsors of this event.}
\end{frame}

\begin{frame}{The Problem}
  \begin{enumerate}
    \item What can we learn about 1Password user's behavior without putting them at risk?
    \item What technologies for doing so are within our reach?
  \end{enumerate}
\end{frame}

\subsection{The example behavior question}

\begin{frame}{Starting strong}
  \begin{itemize}
    \item If we can figure this out for extremely sensitive information then we can do it for anything.
    \item I pick a real example question about user behavior that involves understanding extremely sensitive data.
  \end{itemize}
\end{frame}

\begin{frame}{Some speculation}
  \blockcquote{Goldberg18:MFA:pwc}{The use of MFA [for 1Password itself] may lead people to use weaker [account] passwords, thereby strengthening a less crucial part of their security (authentication), while weakening a far more important component.}
\end{frame}

\begin{frame}{The example question}
  \begin{block}{Data question}
    Is there a negative correlation between use of 2FA for 1Password itself and the strength of a the account password?
  \end{block}

\end{frame}

\begin{frame}{What we know (and don't)}
  \begin{enumerate}
    \item We know who has 2FA switched on. (This is necessary to provide the service.)
    \item We don't know the password strength of anybody's account password.
  \end{enumerate}

\end{frame}

\begin{frame}{What we never want to know}
  \begin{alertblock}{For user eyes only}
    We don't ever want to know the strength of anyone's account password.
  \end{alertblock}
\end{frame}

\begin{frame}{Reasons to not know}
  (I don't think I really need to list the reasons.)
\end{frame}

\begin{frame}{Not knowing in not enough}
  Not knowing is not enough. The world needs to know that we don't know.
  \begin{itemize}
    \item Each participating user should be able to determine that their privacy is being protected.
    \item The Security and Privacy communities must be able to confirm that our system behaves as we say it does.
  \end{itemize}

\end{frame}

\begin{frame}{Today's talk}
  Today's talk is an over view of what I have learned so far about how we might do this.

  Much of what I learned is new or new-ish to me. But there is nothing fundamentally new here. So you can spend the remainder of this talk napping.

\end{frame}

\section{Overview of potential techniques}

\begin{frame}{General structure}
  \begin{itemize}
    \item Clients have the sensitive data.
    \item Clients do something to that data and send it to a system that is not under the user's control.
  \end{itemize}
\end{frame}

\subsection{Anonymization}

\begin{frame}{Removing identifers}
  \begin{itemize}
    \item Clients will not send any identifying data.
    \item Clients can't avoid “sending” IP addresses. (Tor might help with that)
  \end{itemize}
\end{frame}

\begin{frame}{De-Anonymization is a thing}
  \begin{itemize}
    \item Anonymization is hard, but even when done right it isn't enough.
    \item Information wants to be free! (Assume your data will leak)
    \item Anonymized data combined with other (public) data can be de-anonymized.
    \item That other public data can come from sources you have nothing to do with.
    \item That other data might not yet exist at the time you create your scheme.
  \end{itemize}
\end{frame}

\begin{frame}{Netflix case}
  (Talk about Netflix de-anonymization if time)
\end{frame}

\begin{frame}[standout]
  Good anonymization and data protection is necessary. But they are far from sufficient.
\end{frame}

\subsection{Vague response}

\begin{frame}{Wide bins}
  The client shouldn't report password strength with the full precision that it knows,
  but could, say, use three bins: low, medium, high.
\end{frame}

\subsection{Random response}

\begin{frame}{Welcome to the 60s!}
  In one setup~\cite{BBB:JoP79} Subjects given a questionnaire but told to roll a die before answering each question.
  The instructions told them to answer differently depending on their roll of the die
  \begin{itemize}
    \item[1--4] Answer honestly
    \item[5] Answer “yes”
    \item[6] Answer “no”
  \end{itemize}
\end{frame}

\begin{frame}{Known probability of alternative}
  More recently subjects drew a red or green ball out of a sack and answered the red or green question. \cite{LaraETAL06:RRT}
  \begin{itemize}
    \item[Red] “Did you ever interrupt a pregnancy?"
    \item[Green] “Were you born in April?”
  \end{itemize}
  This was conducted in Mexico in 2001. Abortion was highly stigmatized and illegal.
  \note[item]{The study concluded that 16.3\% of women in their sample had had at least one induced abortion (standard error of 0.016).}
\end{frame}

\begin{frame}{It works}
  Researchers have been able to demonstrate that the increased rate of honest responses outweighs the statistical noise.
\end{frame}

\begin{frame}{Is it enough}
  Suppose we set up a system to answer honestly about password strength 50\% of the time.
  If our data leaks and is de-anonymized is the user sufficiently protected.
\end{frame}

\begin{frame}{Too much noise?}
  \begin{itemize}
    \item Vague response reduces the power of statistical tests.
    \item Random response reduces the power of statistical tests.
    \item Opt-in reduces the sample size and introduces a selection bias.
  \end{itemize}
  (I have started to play with simulations to see what kind of samples and parameters are likely to still produce usable results)
\end{frame}

\section{Differential privacy}

\begin{frame}{Differential privacy}
  \begin{itemize}
    \item DP is about limiting the possibility of de-anonymization
    \item DP provide a common mathematical notion of dataset privacy protections across a wide variety of techniques
    \item It adds statistical noise in ways similar to vague and random response
    \item The noise it adds is designed to make it possible to see its effect on privacy
  \end{itemize}
\end{frame}

\begin{frame}{DP is not \dots}
  \begin{itemize}
    \item \dots a single technique
    \item \dots applied at a common point of data processing
    \item \dots particularly easy.
  \end{itemize}
\end{frame}

\begin{frame}{When and where}
  Different DP techniques can be done at
  \begin{itemize}
    \item at data collection time
    \item to transform a dataset
    \item at data analysis time
  \end{itemize}
  \note[item]{The opendp project has a really nice set of tools for generating privacy preserving reports from data sets that are not privacy preserving}
  In our example, we never want to have password strengths so it would have to be done client side at data collection time.
\end{frame}

\subsection{Homomorphic encryption}

\begin{frame}{Example: Millionaire's problem}
  \blockcquote{yao1982protocols}{Two millionaires wish to know who is richer;
    however, they do not want to find out inadvertently any additional information about each other’s wealth. How can they carry out such a conversation?}
  \note[item]{millionaires were different in 1982}
\end{frame}


\begin{frame}{Homomorphic encryption}
  \begin{itemize}
    \item For any protocol that can run in polynomial time using a trusted third party, there is a protocol that can produce the same results without a TTP\@.
    \item Allows multiple parties to compute things over their individual secrets without revealing secrets to each other or a third party.
  \end{itemize}
\end{frame}

\begin{frame}{Practicalities}
  \begin{itemize}
    \item Still need a PhD to make real use of it?
    \item Polynomial time and space doesn't actually mean fast and small.
    \item Protocols have a lot of back and forth.
  \end{itemize}
\end{frame}

\section{A rant on noise}

\begin{frame}{“We can't handle statistical error”}
  \begin{itemize}
    \item These techniques add statistical error
    \item They add known, quantifiable amounts of statistical error
    \item Thus you either need a larger sample, or you have wider confidence intervals in your results
    \item Some objections have been “we can't accept anything with statistical error or confidence intervals."
  \end{itemize}

\end{frame}

\begin{frame}{There is always statistical error}
  These techniques add known amounts of error on top of the error that one would \emph{already have} without these techniques.
\end{frame}

\begin{frame}[standout]
  If you can't handle statistical error or you don't know that you always have some you should not call yourself a data analyst.
  \note[item]{This is why I don't have friends}
\end{frame}

\section{Conclusions?}

\begin{frame}{Can we do it?}
  Is there a combination vagueness, random responses, and data protections on the acquired data
  that would offer sufficient guarantees for our users
  and allow us to answer the 2FA/strength question with sufficient confidence?
\end{frame}

\begin{frame}{I don't know}
  When I started, I hoped my math was good enough to figure this out analytically. Now I must resort to simulations.
\end{frame}

\appendix

\section{Netflix case}
\begin{frame}{Netflix data release}
  \begin{itemize}
    \item In 2006 Netflix publicly released an anonymized dataset of about 100 million movie ratings from about 480,000 subscribers.
    \item From Netflix FAQ “all customer identifying information has been removed; all that remains are ratings and dates.”
  \end{itemize}
\end{frame}

\begin{frame}{IMDB public data}
  \begin{itemize}
    \item Some Netflix subscribers publicly share some movie ratings in other places.
    \item Some of those people may not want to world do know what about some of the other movies they watched.
  \end{itemize}
\end{frame}

\begin{frame}{De-anonymization}
  \Textcite{narayanan2008robust} use Netflix and IMDB data to illustrate a general algorithm for de-anonymization:

  \begin{quotation}
    With 8 movie ratings [from IMDB] (of which 2 may be completely wrong) and dates that may have a 14-day error, 99\% of records [can] be uniquely identified in the dataset. For 68\%, two ratings and dates (with a 3-day error) are sufficient.
  \end{quotation}

\end{frame}

\begin{frame}{Ten years later}{The test of time}

  \begin{quote}
    Attacks only get better with time. The flood of de-ano\-nym\-i\-za\-tion demonstrations in the last decade makes for a strong argument that database privacy should rest on provable guarantees rather than the absence of known attacks. The flourishing research on differential privacy is thus a welcome development. \parencite[p.~1]{narayanan2019robust}
  \end{quote}

  And there have been a flood of de-anonymization demonstrations.

\end{frame}

\section{Resources}

\begin{frame}{Resources}
  \begin{itemize}
    \item Repository containing these slides (and a few other things): \url{\reporoot}
    \item \href{https://opendp.org}{OpenDP}: Really nice tools and community for the DP at the data analysis stage. \url{https://opendp.org}
    \item \href{https://www.usenix.org/conferences/byname/1046}{Usenix PEPR}: Privacy Engineering, Practice and Respect.
  \end{itemize}

\end{frame}


\begin{frame}[t,allowframebreaks]
  \frametitle{References}
  \printbibliography[heading=none]
\end{frame}

%%%%%%%%%%%%%%%%%%%%%%%%%%%%%%%%%%%
%%%%%%%%%%%%%%%%%%%%%%%%%%%%%%%%%%%
\end{document}
